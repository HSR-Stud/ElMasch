\section{Antriebstechnik}
\renewcommand{\arraystretch}{1.1}
\subsection{Gradlinige Bewegungen}
\begin{tabular}[c]{ | p{5cm} | p{8cm} | p{4cm} | }
	\hline
	Kraft & $F=m\cdot a=m\cdot\frac{dv}{dt}$ & N \\
	\hline
	Leistung & $P=v\cdot F$ & W \\
	\hline
	Kinetische Energie & $W_{kin}=\frac{1}{2}\cdot m \cdot v^2$ & Ws \\
	\hline
\end{tabular}

\subsection{Rotierende Bewegungen}
\begin{tabular}[c]{ | p{5cm} | p{8cm} | p{4cm} | }
	\hline
	Winkelgeschwindigkeit & $\omega = 2\cdot\pi\cdot n$ & $s^{-1}$ \\
	\hline
	Geschwindigkeit & $v=\omega \cdot r = 2 \cdot\pi\cdot n \cdot r$ &
	$\frac{m}{s}$
	\\
	\hline
	Leistung & $P=\frac{M}{r}\cdot 2 \cdot\pi\cdot n \cdot r = 2\cdot\pi\cdot n
	\cdot M = \omega \cdot M$ & W \\
	\hline
	Kinetische Energie & $W_{kin}=\frac{1}{2}\cdot J \cdot\omega^2$ & Ws \\
	\hline
\end{tabular}

\subsection{Massenträgheit und Beschleunigung}
\begin{tabular}[c]{ | p{8cm} | p{7cm} | p{2cm} | }
	\hline
	Trägheitsmoment & $J=\int\limits_{0}^mr^2\cdot dm$ & $kgm^2$ \\
	\hline
	Beschleunigungsmoment & $M_B=J\cdot\alpha = J\cdot\frac{d\omega}{dt}=J\cdot
	2\pi\cdot\frac{dn}{dt}$ & Nm\\
	\hline
	Energieaufwand für die Beschleunigung & $W=\frac{1}{2}\cdot J \cdot
	\left(\omega_2^2-\omega_1^2\right) = \frac{1}{2}\cdot
	m\cdot\left(v_2^2-v_1^2\right)$ & Ws\\
	\hline
\end{tabular}

\subsection{Umrechnen der Bewegungsgrössen auf die Motorwelle}
\begin{tabular}[c]{ | p{14.5cm} | p{3cm} | }
	\hline
	Übersetzungsverhältnis des Getriebes & $i=\frac{n_1}{n_2}$\\
	\hline
	Moment der Lastmaschine, auf die Motorwelle umgerechnet &
	$M_1=\frac{M_2}{\eta_G}\cdot\frac{n_2}{n_1}$\\
	\hline
	Trägheitsmoment der Lastmaschine, auf die Motorwelle umgerechnet &
	$J_1=\frac{J_2}{\eta_G\cdot i^2}$\\
	\hline
	Ersatzmoment einer Masse m, die sich gradlinig mit v bewegt, bezogen auf eine
	Welle mit $\omega$ & $M_{ers}=\frac{F\cdot v}{\omega}=F\cdot r_{ers}$\\
	\hline
	Ersatzträgheitsmoment einer Masse m & $J_{ers}=m\cdot r_{ers}^2$ \\
	\hline
\end{tabular}

\subsection{Beschleunigungsvorgänge}
\begin{tabular}[c]{ | p{9cm} | p{8.5cm} | }
	\hline
	Dynamisches Grundgesetz der Antriebstechnik & $J\cdot\frac{d\omega}{dt}= \sum
	M(\omega)=M_{Mot(\omega)}+M_{Last(\omega)}$\\
	\hline
	Stationärer Lastfall & $\sum M=0=M_{Mot}+M_{Last}$\\
	\hline
	Zeitbedarf für eine Drehzahlerhöhung von $\omega_1$ auf $\omega_2$ &
	$t_{an}=J\int\limits_{\omega_1}^{\omega_2}\frac{1}{M_B}\cdot d\omega$ \\
	\hline
	Bei konstantem Beschleunigungsmoment &
	$t_{an}=\frac{J}{M_b}\cdot\left(\omega_2-\omega_1\right)$ \\
	\hline
	Winkelgeschwindigkeit entsprechend Nenndrehzahl $n_N$ & $\omega_N$\\
	\hline
	Bei linear abnehmendem Beschleunigungsmoment &
	$t_{an}=\frac{J}{M_{B_{max}}}\cdot\omega_N\cdot\int\limits_{\omega_1}^{\omega_2}\frac{1}{\omega_N-\omega}\cdot
	d\omega$ \\
	\hline
	Für $\omega_1 < \omega_2 < \omega_N$ &
	$t_{12}=\frac{J}{M_{B_{max}}}\cdot\omega_N\cdot
	ln\frac{\omega_N-\omega_1}{\omega_N-\omega_2}$ \\
	\hline
	Anlaufkonstante & $T_an$\\
	\hline
	Stillstand zur Nenndrehzahl $\omega_N$ & $t{an}\approx 5\cdot T_{an}\approx
	5\cdot\omega_N\cdot\frac{J}{M_{B_{max}}}$\\
	\hline
\end{tabular}

\subsection{Stabilität von Arbeitspunkten}
Der stationäre Betriebspunkt eines Antriebs ist durch den Schnittpunkt der
M-n-Kennlinien von Motor und Antriebsmaschine gekennzeichnet\\
\begin{tabular}{ll}
$M=-M_W$\\
$M$ & Motormoment\\
$M_W$ & Lastmoment\\
\end{tabular}\\
Ein Arbeitspunkt ist stabil, wenn bei einer virtuellen Drehzahlerhöhung das
Widerstandsmoment $M_W$ grösser ist als das Motormoment $M$, d.h. einer
Drehzahlerhöhung entgegenwirkt\\
$$\frac{\Delta M- \Delta M_W}{\Delta \omega}<0$$\\
