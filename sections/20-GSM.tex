\section{Gleichstrommaschinen (GSM)}
    \subsection{Funktionsprinzip}
    Ein Gleichstrom ist durchfliesst ein Schleife, welche sich in einem Magnetfeld befindet. \\
    Durch die Lorenzkraft wirkt auf die Schleife ein Drehmoment, welche sie in eine maximal halbe Drehung versetzt.
    Danach muss man den Schleifenstrom umpolisieren, um die Drehung weiter zu führen. \\
    \abb{images/GSM_Funktion.png}{14cm}{Funktion eines GSM mit Stromwender}
    \abb{images/GSM_Ersatz.png}{14cm}{Erstzschema der 4 Schaltungsarten}
    \textbf{A1-A2:} Ankerwicklung, \textbf{B1-B2:} Wendepolwicklung, \textbf{C1-C2:} Kompensationswicklung, \\
    \textbf{D1-D2:} Reihenschlusswicklung, \textbf{E1-E2:} Nebenschlusswicklung, \textbf{F1-F2:} Fremderregte Wicklung \\
    \begin{figure}
	    \centering
	    \begin{subfigure}[t]{0.45\textwidth}
	    	\centering
	    	\includegraphics[width=\textwidth]{images/GSM_Wicklungen.png}
	    	\caption{Entstehung einer Schleifenwicklung}
	    \end{subfigure}
	    \begin{subfigure}[t]{0.45\textwidth}
	    	\centering
	    	\includegraphics[width=\textwidth]{images/GSM_Drehmomentdarstellung.png}
	    	\caption{Alle Drehmomente zusammen ergeben ein gleichmässiges Drehmoment}
	    \end{subfigure}
    \end{figure}

    
    \subsection{Grundgleichungen}
    \begin{minipage}{15.1cm}
    \begin{tabular}[c]{ | p{6cm} | p{8cm} |}
    	\hline
    	Umfangsgeschwindigkeit & v$_u=\omega\cdot R = 2\pi\cdot n \cdot
    	R=\frac{2\pi\cdot f \cdot R}{p}=2\cdot f \cdot \tau_p = d\cdot\pi\cdot n$\\
    	 & Geometrisch: v$=d\cdot\pi\cdot f$\\
    	\hline
    	Teilspannung eines Leiters & $U_{iL}=2\cdot \tau_p \cdot f \cdot l \cdot
    	B_m= 2\cdot f\cdot \Phi = 2\cdot p \cdot n \cdot \Phi$\\
    	\hline
    	Gesamtspannung & $U_i=\frac{z}{2\cdot a}\cdot U_{iL}=\frac{z}{a}\cdot
    	p \cdot n \cdot \Phi=k_1\cdot\Phi\cdot n = B_m \cdot z \cdot l \cdot v$\\
    	\hline
    	Ankerspannung & $U_a=R_a\cdot I_a + L_a\frac{dI_a}{dt}+U_i = U_i + U_{BK} + I_{AN} \cdot R_A$ \\
    	\hline
    	Erregerspannung & $U_e=R_e\cdot I_e + L_e\frac{dI_e}{dt}$\\
    	\hline
    	Induzierte Spannung & $U_i = k_1\cdot \Phi \cdot \omega_{mech} = B\cdot l
    	\cdot $v$ \quad k_1 = Maschinenkonstante$\\
    	\hline
    	Elektrisch verursachtes Drehmoment & $M_{el}=M_{Welle}+M_{Reibung}+J\cdot\frac{d\omega_{mech}}{dt}=k_1\cdot
    	\Phi\cdot I_a$\\
    	\hline
    	Erregerfluss & $\Phi = \frac{L_e}{N_e}\cdot I_e$\\
    	\hline
    	Ideale Luftspaltleistung & 
    		$P_{Luft}=U_i\cdot I_A = M_{Luft} \cdot \omega $\\
    	\hline
    	Ideales Luftspaltdrehmoment & 
    		$M_{Luft} = N \cdot B \cdot I \cdot d \cdot l$ \\
    	\hline
    	Mech. Leistung an der Welle &
    	$P_{Welle}=P_{Luft}-V_{Fe}-V_{Reib}-V_{zus}$ \quad $V{...} =$ Verluste\\
    	\hline
    	Nenndrehmoment $M_{N}$ & $\frac{P_{Mech}}{2 \cdot \pi n_{0}}$ \qquad Vorsicht $n_0$ nicht in $min^{-1}$!! \\ \hline
    	Ankerwiderstand $R_{A}$ & $R_{A} = \frac{U_{N} - U_{iN}}{I_{N}} $ \qquad $U_{iN} =$ Induzierte Spannung \\ \hline
    \end{tabular}
    \end{minipage}
    \begin{minipage}{5cm}
    \begin{tabular}{|p{4cm}|}
    \hline
	$U_I = U_{Ind} =$ Induzierte Spannung \\
	\hline
	$U_A =$ Ankerspannung\\
	\hline
	$U_E =$ Erregerspannung \\
	\hline
	$k_1 , k_2 =$ Maschinenkonstanten \\
	\hline
	$P_{Luft} = P_{Mech} =$ Mechanische Leistung \\
	\hline
	$n_0$ oder $n_{N0} =$ Leerlaufdrehzahl \\
	\hline
	$n_N =$ Nenndrehzahl \\
	\hline
	
	
	\end{tabular}
	\end{minipage}
    
    	
    
    \subsection{Nebenschlusmaschinen}
    	Nebenschlussmaschinen am starren Netz unterscheiden sich im Betriebsverhalten nicht von fremderregten Maschinen.
    	
    	
        \renewcommand{\arraystretch}{1.5}
        \begin{tabular}[c]{ | p{6cm} | p{11cm} |}
            \hline
            Drehzahl &
            $ n= \frac{U_A}{k_1 \cdot \Phi} - \frac{R_A \cdot M}{k_1 \cdot k_2 \cdot \Phi^2} \sim \frac{U_A}{I_E} - \frac{M}{I_E^2}$ \newline
            $\Longrightarrow $ bei $I_E$ = 0 und $M$ = 0 $n \rightarrow \infty$ \newline 
            $k_1$, $k_2$ sind Maschinenkonstanten \\
            \hline
            Drehmoment &
            $M=\frac{k_2 \cdot \Phi \cdot U_{Anker}}{R_{Anker}} - \frac{k_1 \cdot k_2 \cdot\phi^2 \cdot n}{R_A} =\frac{P_{mech}}{2\pi n_N}=\frac{U_{iN}I_N}{2\pi n_N}\sim I_E \cdot U_A - I_E ^2 \cdot n$
            \quad Achtung: $[n] = min^{-1}$\\
            \hline
            Leistung &
            $P_{Anker}= U_i \cdot I \pm I^2 \cdot R_A $(+ bei Motor; - bei Generator) \\
            \hline
            Anlaufmoment &
            $M_A = \frac{k_1 \cdot \Phi \cdot U}{R_1}$ \\
            \hline
            Magnetischer Fluss des Hauptfeldes & 
            $\Phi=\frac{L_e}{N_e}\cdot\frac{U_a}{R_e+R_v}$\\
            \hline
            Leerlaufdrehzahl &
            $\omega_m=\frac{N_e\left(R_E+R_V\right)}{k_1\cdot
            L_e}-\frac{R_a\left(R_e+R_V\right)^2N_e^2}{\left(k_1L_eU_a\right)^2}\cdot
            M_{el}$\\
            & $n_0 = \frac{U_A}{k_1 \cdot \Phi}$ \\
            & $n_0 = \frac{n_N \cdot U_{Ind,neu}}{U_{Ind,nenn}}$  wenn Verluste $ \approx$ const. !\\
            & $n_0$ ist Näherungsweise nicht Abhängig von $U_A$ \\
            \hline
            
        \end{tabular}
        \renewcommand{\arraystretch}{1.3}
        
    \subsection{Reihenschluss Maschine}
    Auch bekannt als Seriemaschine oder Hauptschlussmaschine\\
        \renewcommand{\arraystretch}{2}
        \begin{tabular}[c]{ | p{6cm} | p{9cm} |}
            \hline
            Drehmoment &
            $ M=  \frac{U_{Induziert} \cdot I_A}{2 \cdot \pi n_N} = \frac{k_1 \cdot k_E \cdot I_A ^2}{2\cdot \pi n_N}= \frac{k_1 \cdot k_E}{2 \cdot \pi n_N}\cdot(\frac{U}{R_A + R_E + k_1 \cdot k_E \cdot n})^2$ \\
            \hline
            Anlaufstrom &
            $I_A=\frac{U_N}{\sum R_a}=\frac{U_N I_N^2}{U_N I-P_N}$ \\
            \hline
            Elektrisch verursachtes Drehmoment & $M_{el}=\frac{L_e}{N_e}k_1\cdot I_a^2$\\
            \hline
            Leerlaufdrehzahl &
            $\omega_m=\frac{N_e}{L_e}\frac{U_a}{k_1I_a}-\frac{N_e}{L_e}\frac{R_a+R_e}{k_1}=\frac{\sqrt{N_e}\cdot
            U_a}{\sqrt{L_ek_1M_{el}}}-\frac{N_e}{L_e}\frac{R_a+R_e}{k_1}$\\
            \hline
            Anlaufdrehmoment bei begrenztem Anlaufstrom & $M^*_A = (\frac{I^*_A}{I_N})^2 \cdot M_N$ \\ \hline
        \end{tabular}
        \renewcommand{\arraystretch}{1.5}