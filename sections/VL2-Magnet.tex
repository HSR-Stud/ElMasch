\subsection{Magnetische Durchflutung}
\begin{minipage}{0.2 \linewidth}
    \includegraphics[width =\linewidth]{./Pics/VL2/magDurchflutung}
\end{minipage}
\begin{minipage}{0.8 \linewidth}
\textbf{Die magnetische Durchflutung} einer von der Kontur (C) umrandetetn Fläche (A) ist der gesamte Strom, der durch diese Fläche fliesst: \\

$\Theta = \sum_{k = 1}^{n} I_k$\\

Nach dem \textbf{Durchflutungsgesetz} muss das Linienintegral des magnetischen Felds entlang der Kontur (C) die entsprechende elektrische Durchlutung ergeben: \\

$\oint \vec{H} \cdot \vec{dl} = \sum_{k=1}^{n} I_k = \Theta$\\

Diese Gleichung ist eine sehr wichtige Verknüpfung zwischen dem Magnetfeld und dem elektrischen Strom. Sie wird sehr oft \textbf{für die Berechnung des Magnetfelds} einer angegebenen Erreger-Anordnung eingesetzt. 
\end{minipage}
\subsection{Magnetische Spannung}
\begin{minipage}{0.2 \linewidth}
    \includegraphics[width =\linewidth]{./Pics/VL2/magDurchflutung}
\end{minipage}
\begin{minipage}{0.8 \linewidth}
Um das Linienintegral einfacher zu rechnen sollte die Integrationskurve (C) passen definiert und aufgeteilt werden: \\

$\oint_{(C)} \vec{H} cdot \vec{dl} = \oint_{P_1}^{P_2} \vec{H} \cdot \vec{dl} + \oint_{P_2}^{P_3} \vec{H} \cdot \vec{dl}  + \cdots +\oint_{P_{m-1}}^{P_m} \vec{H} \cdot \vec{dl} $ \\

Der Definition der elektrischen Spannung zufolge, wird die magnetische Spannung wie folgt definiert: \\

$V_m = \int^B_A \vec{H} \cdot \vec{dl}$ \\

Die Aufteilung der Integrationskurve(C) wird normalerweise so durchgeführt, dass eine Summation anstatt der Integration ergeben wird: \\

$\oint_{(C)} \vec{H} \cdot \vec{dl} = H_1 \cdot l_1 + H_2 \cdot l_2 + H_3 \cdot l_3 + \cdots = V_{m1} + V_{m2} + V_{m3} + \cdots$
\end{minipage}

\subsection{Magnetischer Widerstand}
\begin{minipage}{0.2 \linewidth}
    \includegraphics[width =\linewidth]{./Pics/VL2/magR}
\end{minipage}
\begin{minipage}{0.8 \linewidth}
Der magnetische Fluss durch einen Luftspalt zwischen den Eisenblöcken lässt sich wie folgt angeben:\\

$\Phi_{1Eisen} = \Phi_{Luft} = \Phi_{2Eisen} \Rightarrow B_{1Eisen} =  B_{Luft} = B_{2Eisen}$ \\

Jedoch ist die magnetische Permeabilität von Eisen etwa 1000 mal grösser als die Permeabilität von Luft, was in dieser Anordnung das magnetische Feld im Luftspalt bestimmt: \\

$B_{1Eisen} = B_{Luft} \Rightarrow \mu_{0} \cdot \mu_{rEisen} \cdot H_{Eisen} = \mu_0 \cdot \mu_{rLuft} \cdot H_{Luft}$ \\

$H_{Luft} = \frac{\mu_{rEisen}}{\mu_{rLuft}} \cdot H_{Eisen} \Rightarrow H_{Luft} \approx 1'000 \cdot H_{Eisen}$ \\

Offenbar ist das magnetische Feld in der Luft viel grösser als das entsprechende Feld im Eisen. Das bedeutet, dass die magnetische Spannung entlang einer Flusslinie mit den Luftstrecken der Linie bestimmt wird: \\

$\oint_{(C)} \vec{H} \cdot \vec{dl} = H_{Eisen} \cdot l_{Eisen} + H_{Luft} \cdot l_{Luft} \approx H_{Luft} \cdot l_{Luft} = V_{mLuft}$ \\

Die magnetische Spannung des Luftspalts spielt eine wichtige Rolle in jeder Anordnung mit einem Eisenkern: \\

$\oint_{(C)} \vec{H} \cdot \vec{dl} \approx H_{Luft} \cdot l_{Luft} = V_{mLuft}$ \\

Der Definition des elektrischen Widerstands zufolge, wird den magnetischen Widerstand wie folgt definiert: \\

$R_m = \frac{V_m}{\Phi}$ \\

Wenn $A$ die auf den magnetischen Fluss senkrechte Fläche des Luftspaltes ist, lässt sich denn der magnetische Widerstand des Luftspaltes wie folgt angeben: \\

$R_m = \frac{H_0 \cdot l}{B_0 \cdot A} = \frac{H_0 \cdot l}{\mu_0 \cdot H_0 \cdot A} = \frac{l}{\mu_0 \cdot A}$
 \end{minipage}

\subsection{Magnetkreis}

Die Annahme der Feldberechnung: \\
Das magnetische Feld in der Spule ist \textbf{viel grösser} als draussen und das Feld in der Spule \textbf{ist homogen}. Die Annahmen sind dann realistisch, wenn $L >> R$ ist.\\

$\oint_{(C)} \vec{H} \cdot \vec{dl} \approx H \cdot L = N \cdot I \Rightarrow H = \frac{N \cdot I}{L}, B = \mu_0 \cdot \frac{N \cdot I}{L}$

\begin{minipage}{0.5 \linewidth}
\includegraphics[width = \linewidth]{./Pics/VL2/spule}
\end{minipage}
\begin{minipage}{0.5 \linewidth}
Das magnetische Feld und die magnetische Flussdichte der Spule: \\

$H = \frac{N \cdot I}{L}$ \\

Für die Spule mit den folgenden geometrischen Daten: \\

R = 0.1m, L = 0.27m, N = 10, I = 1A \\

wird das folgende magnetische Feld gerechnet: \\

$H = 37.4 \frac{A}{m}, B = 46.5 \mu T $\\

$\mu_0 = 4 \pi \cdot 10^{-1}$ ($\frac{Tm}{A}$)
\end{minipage}

\begin{minipage}{0.4 \linewidth}
\includegraphics[width = \linewidth]{./Pics/VL2/magKreis}
\end{minipage}
\begin{minipage}{0.6 \linewidth}
Gemäss dem Durchflutungsgesetz lässt sich das Magnetfeld eines Magnetkreises wie folgt angeben: \\

$\oint_{(C)} \vec{H} \cdot \vec{dl} = H_{Fe} \cdot l_{Fe} + 2 \cdot \delta \cdot H_{\delta}  = I \cdot N $ \\

Wie schon präsentiert, ist in dieser Anordnung das magnetische Feld im Luftspalt viel grösser als das entsprechende Feld im Eisen: \\

$H_{\delta} \approx \cdot H_{Fe} \Rightarrow H_{\delta} = \frac{I \cdot N}{2 \delta} $ \\

$B_{\delta} = \mu_0 \frac{I \cdot N}{2 \delta} $ \\

Ein Magnetkreis mit einer stromdurchflossenen Spule erzeugt einen magnetischen Fluss, der den entsprechenden Magnetfluss ohne Magnetkreis mehrere Grössenordnungen überschiesst. 
\end{minipage}

\subsubsection{Beispiele}
\includegraphics[width = \linewidth]{./Pics/VL2/magKreisBeispiele}

\subsection{Permanentmagnet}

\begin{minipage}{0.3 \linewidth}
\includegraphics[width = \linewidth]{./Pics/VL2/permanentMagnet}
\end{minipage}
\begin{minipage}{0.7 \linewidth}
Ein Permanentmagnet (Dauermagnet) ist ein Stück eines magnetisierbaren Materials (Eisen, Kobalt, Nickel, Ferrit) welches sein statisches Magnetfeld behält ohne einen elektrischen Stromfluss. \\

Kennwerte der Dauermagnete:
\begin{description}
\item{Koerzitivfeldstärke $H_c$} Dieses Magnetfeld muss erzeugt werden um den Dauermagneten vollständig zu entmagnetisieren (B=0)
\item{Remanenz $B_R$} Das ist die magnetische Flussdichte des Dauermagnets ohne magnetisches Fremdfeld (H=0)
\end{description}

Die Permanentmagneten ersetzen die Erregerspulen in den elektrischen Maschinen. Damit sind die Materialkosten und die ohmschen Verluste der Erregerspulen eliminiert. 
\end{minipage}

\subsection{Reluktanzkraft}
\begin{minipage}{0.4 \linewidth}
\includegraphics[width = \linewidth]{./Pics/VL2/reluktanz}
\end{minipage}
\begin{minipage}{0.6 \linewidth}
Die ferromagnetischen Körper sind im magnetischen Fremdfeld der so genannten Reluktanzkraft ausgesetzt. \\

Die Reluktanzkraft wirkt auf die ferromagnetischen Körper nur anziehen. \\

In dieser Anordnung lässt sich die Reluktanzkraft so angeben: \\

$F_R  = \mu_0 \frac{N^2 \cdot I^2 \cdot A}{4 \delta^2}$ \\

wobei A die magnetisch wirksame Fläche des Luftspalts ist.
\end{minipage}

\subsection{Zusammenfassung}

\begin{itemize}
\item Das magnetische Durchflutungsgesetz wird häufig für die Magnetfeldberechnung der stromdurchflossenen Spulen mit und ohne Magnetkreis eingesetzt. 
\item Ein Magnetkreis mit einer stromdurchflossenen Spule erzeugt einen magnetischen Fluss, der den entsprechenden Magnetfluss ohne Magnetkreis mehrere Grössenordnungen überschiesst.
\item Die magnetische Stromkraft wirkt auf einen stromführenden Leiter in einem fremden Magnetfeld.
\item Die magnetische Reluktanzkraft wirkt auf einen ferromagnetischen Körper im fremden Magnetfeld.
\item Die Permanentmagneten ersetzten stromdurchflossene Erregerspulen in den elektrischen Maschinen. Damit werden die Materialkosten und die ohmschen Verluste der Errgerspule eliminiert.
\end{itemize}